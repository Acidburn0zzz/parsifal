\documentclass{beamer}
\usepackage[latin1]{inputenc}
\usepackage[T1]{fontenc}
\usepackage{listings}
\usepackage{array}
\lstset{language=Caml, basicstyle=\footnotesize,
  morekeywords={enum, alias, struct, union,
                asn1_alias, asn1_union, asn1_struct,
                UnknownVal, with_lwt, Exception}}


\title{Parsifal: a tutorial}
\author{Olivier Levillain}
\date{}

\begin{document}


%%%%%%%%%%%%%%%%%%%%%%%%%%%%%%%%%%%%%%%%%%%%%%%%%%%%%%%%%%%%%%%%%%%%%%%%%
\begin{frame}
  \maketitle
\end{frame}

\begin{frame}{Parsifal}
  This tutorial is about Parsifal: a generic framework to write binary
  parsers in OCaml.

  \bigskip

  After a short introduction, the slides explain how to install and
  build Parsifal.

  \bigskip

  The remaining of the presentation are step-by-step implementation of
  some toy parsers (TAR, DNS).
\end{frame}

\begin{frame}{Outline}
  \tableofcontents
\end{frame}



%%%%%%%%%%%%%%%%%%%%%%%%%%%%%%%%%%%%%%%%%%%%%%%%%%%%%%%%%%%%%%%%%%%%%%%%%
\section{Context}

\begin{frame}{Outline}
  \tableofcontents[currentsection]
\end{frame}

\begin{frame}{Starting point of our work: SSL data (1/2)}
  Several campaigns to collect SSL data, between July 2010 and July
  2011, using the following methodology:
  \begin{itemize}
  \item enumerating IPv4 hosts with 443/tcp open
  \item sending \texttt{ClientHello} messages
  \item recording the server answer
  \end{itemize}

  \bigskip

  \uncover<2->{Using 10 such campaigns, we analysed several
    parameters:
    \begin{itemize}
    \item TLS parameters
    \item ceritification chain quality
    \item server behaviour against different stimuli
    \item results published at ACSAC 2012
    \end{itemize}
  }
\end{frame}

\begin{frame}{Starting point of our work: SSL data (2/2)}
  Our goal was to extract from those 140~GB of data
  \begin{itemize}
  \item relevant infomration (the messages and certificates received)
  \item quickly
  \item in a robust way
  \end{itemize}

  \bigskip

  \uncover<2->{The Electronic Frontier Foundation, which published
    part of the data we used, also provided some analyses
    \begin{itemize}
    \item they mostly used standard tools (\texttt{openssl})
    \item they mostly focus on certificates
    \end{itemize}
  }
\end{frame}

\begin{frame}{Some history of our parsers (1/2)}
  To handle this amount of data, we used custom tools; several
  implementaions were developped
  \begin{itemize}
  \item<2-> a first Python prototype: quick to write, slow to run
  \item<3-> a second version in C++ (using templates and objects):
    \begin{itemize}
    \item rather extensible (thanks to a description language)
    \item faster than Python, but
    \item hard to debug (memory leaks, segfaults)
    \item long to write (each new feature required too much code)
    \end{itemize}
  \item<4-> third version in OCaml (using a Domain Specific Language resembling Python):
    \begin{itemize}
    \item fast and extensible
    \item more robust than the previous one
    \item but still too much code to write
    \end{itemize}
  \end{itemize}
\end{frame}

\begin{frame}{Some history of our parsers (2/2)}
  \begin{itemize}
  \item the last version, still in OCaml, is called Parsifal
  \item<2-> it relies on a pre-processor to automate most of the tedious steps
    \begin{itemize}
    \item the resulting code is fast, robust and concise
    \end{itemize}
  \end{itemize}

  \bigskip

  \uncover<2->{Code is available on GitHub (\texttt{ANSSI-FR/parsifal})}
\end{frame}


\begin{frame}{Parsifal in a nutshell}
  \begin{itemize}
  \item Generic framework to write \textbf{concise} parsers
  \item \textbf{Speed} of the produced programs
  \item \textbf{Robustness} of the developped tools
  \item Development methodology adapted to write parsers
    \textbf{incrementally}\\[.5cm]

  \item<2-> Parsifal also allows to dump the described objects
  \item<2-> Example: a DNS client in 200~loc\\[.5cm]

  \item<3-> Parsifal main goals
    \begin{itemize}
    \item trusted analysis tools (SSL, X.509, Kerberos, OpenPGP...)
    \item basic blocks to sanitize files or protocol messages (PNG,
      PKCS\#10 CSR...)
    \end{itemize}
  \end{itemize}
\end{frame}



%%%%%%%%%%%%%%%%%%%%%%%%%%%%%%%%%%%%%%%%%%%%%%%%%%%%%%%%%%%%%%%%%%%%%%%%%
\section{Installation}

\begin{frame}{Outline}
  \tableofcontents[currentsection]
\end{frame}

\begin{frame}{OCaml installation}
  \begin{block}{}
    \tt
    \begin{tabular}{l}
      apt-get install ocaml ocaml-findlib \\
      apt-get install liblwt-ocaml-dev \\
      apt-get install libcryptokit-ocaml-dev \\
      apt-get install libounit-ocaml-dev \\
      apt-get install make \\
      \\
      apt-get install git \\
    \end{tabular}
  \end{block}

  \bigskip

  \textit{(Tested on Debian Wheezy)}
\end{frame}

\begin{frame}{Parsifal compilation and installation}
  \begin{block}{Cloning git repository}
    \tt
    \begin{tabular}{l}
      git clone https://github.com/ANSSI-FR/parsifal.git \\
      cd parsifal \\
    \end{tabular}
  \end{block}

  \begin{block}{Compilation}
    \tt
    \begin{tabular}{l} 
      make \\
      LIBDIR=\$HOME/.ocamlpath BINDIR=\$HOME/bin make install \\
      export OCAMLPATH=\$HOME/.ocamlpath \\
      PATH=\$HOME/bin:\$PATH \\
    \end{tabular}
  \end{block}

  \bigskip

  \textit{(Without \texttt{LIBDIR} or \texttt{BINDIR}, system dirs are
    used)}
\end{frame}

\begin{frame}{How to create a project}
  \begin{block}{First project}
    \tt
    \begin{tabular}{l}
      ./mk\_project.sh helloworld \\
      cd helloworld \\
      make \\
      ./helloworld \\
    \end{tabular}
  \end{block}

  \bigskip

  \begin{block}{}
    The project created simply display \texttt{Hello, world!}, but
    uses a Makefile using \texttt{parsifal\_syntax}, the preprocessor.

    \medskip

    Let us now discover the constructions provided by Parsifal.
  \end{block}
\end{frame}



%%%%%%%%%%%%%%%%%%%%%%%%%%%%%%%%%%%%%%%%%%%%%%%%%%%%%%%%%%%%%%%%%%%%%%%%%
\section{Constructions}

\begin{frame}{Outline}
  \tableofcontents[currentsection]
\end{frame}


\begin{frame}[fragile]{Principle}
  The idea is to let the developper write short type descrptions, and
  to expand them automatically to obtain the following elements:
  \begin{itemize}
  \item an OCaml type \texttt{t}
  \item a parsing function \texttt{parse\_t}
  \item a dumping function \texttt{dump\_t}
  \item a function to convert \texttt{t} to a printable value
    \texttt{value\_of\_t}
  \end{itemize}

  \medskip

  \begin{block}{Function prototypes}
    \begin{lstlisting}
      parse_t : string_input -> t
      dump_t : POutput.t -> t -> unit
      value_of_t : t -> value
    \end{lstlisting}
  \end{block}
\end{frame}

\begin{frame}{PTypes}
  Such an enriched type (an OCaml type accompagnied by the thhree
  functions.

  \bigskip

  There are three kinds of PTypes:
  \begin{itemize}
  \item basic PTypes (integers, strings, lists, etc.) are provided by
    the Parsifal core library
  \item keyword-assisted PTypes, described in the following slides
  \item custom PTypes can be written manually
  \end{itemize}
\end{frame}

\begin{frame}[fragile]{Enumerations}
  \textbf{Goal}: use a sum type resembling C enums with strong types

  \bigskip

TLS versions are encoded by a 2-byte (16-bit) value:
{\footnotesize
\begin{lstlisting}
enum tls_version (16, UnknownVal V_Unknown) =
  | 0x0002 -> V_SSLv2, "SSLv2"
  | 0x0300 -> V_SSLv3, "SSLv3"
  | 0x0301 -> V_TLSv1, "TLSv1.0"
  | 0x0302 -> V_TLSv1_1, "TLSv1.1"
  | 0x0303 -> V_TLSv1_2, "TLSv1.2"
\end{lstlisting}
}

  \bigskip

  Enums also come with useful extra functions:
  \begin{itemize}
  \item \texttt{int\_of\_tls\_version}
  \item \texttt{string\_of\_tls\_version}
  \item \texttt{tls\_version\_of\_int}
  \item \texttt{tls\_version\_of\_string}
  \end{itemize}
\end{frame}


\begin{frame}[fragile]{Structures (1/2)}
  \textbf{Goal}: handle a sequence of fields

  \bigskip

  For example, TLS alerts simply are a structure containing two 1-byte
  fields, described as follows in RFC 5246 (TLSv1.2):
  {\footnotesize
\begin{lstlisting}
enum { warning(1), fatal(2), (255) } AlertLevel;

enum {
    close_notify(0),
    unexpected_message(10),
     ...
    unsupported_extension(110),
    (255)
} AlertDescription;

struct {
    AlertLevel level;
    AlertDescription description;
} Alert;
\end{lstlisting}
}

\end{frame}


\begin{frame}[fragile]{Structures (2/2)}
  \bigskip

TLS alerts in Parsifal:
{\footnotesize
\begin{lstlisting}
enum tls_alert_level (8, UnknownVal AL_Unknown) =
  | 1 -> AL_Warning
  | 2 -> AL_Fatal

enum tls_alert_type (8, UnknownVal AT_Unknown) =
  | 0 -> AT_CloseNotify
  | 10 -> AT_UnexpectedMessage
    ...
  | 110 -> AT_UnsupportedExtension

struct tls_alert =
{
  alert_level : tls_alert_level;
  alert_type : tls_alert_type
}
\end{lstlisting}
}

\end{frame}


\begin{frame}[fragile]{Unions}
  \textbf{Goal}: create a type depending on a discriminating value

  \bigskip

{\footnotesize
\begin{lstlisting}
union autonomous_system [enrich] (UnparsedAS) =
  | 16 -> AS16 of uint16
  | 32 -> AS32 of uint32

struct bgp_as_path_segment [param as_size] =
{
  path_segment_type : uint8;
  path_segment_length : uint8;
  path_segment_value : list(path_segment_length) of
                       autonomous_system(as_size)
}
\end{lstlisting}
}

\end{frame}


\begin{frame}[fragile]{Alias}
  \textbf{Goal}: aliases allow for renaming PTypes or ASN.1 structures

  \bigskip

{\footnotesize
\begin{lstlisting}
alias ustar_magic = magic["ustar"]
alias tar_file = list of tar_entry

struct atv_content = {
  attributeType : der_oid;
  attributeValue : der_object
}
asn1_alias atv
asn1_alias rdn = set_of atv
asn1_alias distinguishedName = seq_of rdn
\end{lstlisting}
}

\end{frame}

\begin{frame}{Other constructions}
  \begin{itemize}
  \item PContainers, allowing transparent transformations at parsing
    and/or dumping time
    \begin{itemize}
    \item encoding: hexadecimal, base64
    \item compression: DEFLATE, zLib or gzip containers
    \item safe parsing: some containers provide a fall-back strategy
      when the contained PType can not be parsed
    \item miscellaneous checks: CRC, length-checking\\[.5cm]
    \end{itemize}

  \item \texttt{asn1\_union} and \texttt{asn1\_struct}
  \item bit fields
  \end{itemize}
\end{frame}




%%%%%%%%%%%%%%%%%%%%%%%%%%%%%%%%%%%%%%%%%%%%%%%%%%%%%%%%%%%%%%%%%%%%%%%%%
\section{Step by step TAR}

\begin{frame}{Outline}
  \tableofcontents[currentsection]
\end{frame}


\begin{frame}{The TAR format (1/3)}
  A TAR file is composed of entries:
  \begin{center}
    \scriptsize

    \begin{tabular}{|r|r|l|}
      \hline
      \multicolumn{1}{|c|}{\bf Offset} &
      \multicolumn{1}{c|}{\bf Len} &
      \multicolumn{1}{c|}{\bf Description} \\
      \hline

      \hline
      0 & 512 & TAR header, padded with zero bytes \\
      \hline
      512 & \multicolumn{1}{c|}{file size} &
      file content, padded with zero bytes \\
      & \multicolumn{1}{c|}{aligned at the 512-byte boundary)} & \\
      \hline
    \end{tabular}
  \end{center}
\end{frame}


\begin{frame}{The TAR format (2/3)}
  TAR header:
  \begin{center}
    \scriptsize

    \begin{tabular}{|r|r|l|l|}
      \hline
      \multicolumn{1}{|c|}{\bf Offset} &
      \multicolumn{1}{c|}{\bf Len} &
      \multicolumn{2}{c|}{\bf Description} \\
      & & \multicolumn{1}{c|}{\bf TAR} &
      \multicolumn{1}{c|}{\bf \tt ustar} \\
      \hline

      \hline
        0 & 100 & \multicolumn{2}{l|}{Filename} \\
      \hline
      100 &   8 & \multicolumn{2}{l|}{Permissions} \\
      \hline
      108 &   8 & \multicolumn{2}{l|}{UID} \\
      \hline
      116 &   8 & \multicolumn{2}{l|}{GID} \\
      \hline
      124 &  12 & \multicolumn{2}{l|}{File size} \\
      \hline
      136 &  12 & \multicolumn{2}{l|}{Timestamp} \\
      \hline
      148 &   8 & \multicolumn{2}{l|}{Header checksum} \\
      \hline
      156 &   1 & Link type & File type \\
      \hline
      157 & 100 & \multicolumn{2}{l|}{Linked file} \\
      \hline
      257 &   5	& \multicolumn{1}{c|}{-} & \texttt{"ustar"} marker \\
      \hline
      263 &   3	& \multicolumn{1}{c|}{-} & \texttt{ustar} version \\
      \hline
      265 &  32 & \multicolumn{1}{c|}{-} & Owner \\
      \hline
      297 &  32 & \multicolumn{1}{c|}{-} & Group \\
      \hline
      329 &   8 & \multicolumn{1}{c|}{-} & Device major \\
      \hline
      337 &   8 & \multicolumn{1}{c|}{-} & Device minor \\
      \hline
      345 & 155 & \multicolumn{1}{c|}{-} & Prefix \\
      \hline
    \end{tabular}
  \end{center}
\end{frame}

\begin{frame}{The TAR format (3/3)}
  Link type/File type values:
  \begin{center}
    \scriptsize
    \begin{tabular}{|>{\tt}c|l|c|}
      \hline
      \multicolumn{1}{|c|}{\bf Character} &
      \multicolumn{1}{c|}{\bf Description} &
      \multicolumn{1}{c|}{\bf {\tt ustar}-specific} \\
      \hline

      \hline
      <NUL>, 0 & regular file & - \\
      1 & hard link & - \\
      2 & symbolic link & - \\
      \hline
      3 & character device & yes \\
      4 & block device & yes \\
      5 & directory & yes \\
      6 & FIFO & yes \\
      7 & contiguous file & yes \\
      \hline
    \end{tabular}
  \end{center}
\end{frame}


\begin{frame}{TAR v1}
  \texttt{tar-steps/tar1.ml} describe a primitive version of the TAR
  file format:
  \begin{itemize}
  \item a \texttt{file\_type} \texttt{enum} to represent the different
    values
  \item a \texttt{struct} type to describe the complete \texttt{ustar}
    header
  \item in TAR, integer as encoded as a string representing an octal
    value; to decode the \texttt{file\_size} field, the file contains
    a \texttt{int\_of\_tarstring} function
  \item then, a second \texttt{struct} to describe a TAR entry
  \item finally, the \texttt{main} program opens a file and prints the
    names of the files contained in the archive
  \end{itemize}
\end{frame}

\begin{frame}{TAR v2}
  \texttt{tar-steps/tar2.ml} allows the \texttt{ustar} header to be
  optional:
  \begin{itemize}
  \item the \texttt{tar\_header} is now terminated by a
    \texttt{string} field (which corresponds to the remaining string)
    and encapsulated inside a 512-byte container
  \item the \texttt{ustar}-specific part of the header is extracted in
    a new \texttt{struct}
  \item the header includes the \texttt{ustar\_header} field as an
    optional field
  \end{itemize}
\end{frame}

\begin{frame}{TAR v3}
  As explained earlier, integers are encoded as strings representing
  their octal value in TAR archive.

  \medskip

  To handle integers better, we write a custom \texttt{tar\_numstring}
  PType in \texttt{tar-steps/tar3.ml}:
  \begin{itemize}
  \item the \texttt{tar\_numstring} type is \texttt{int}, the
    \textit{intended} value
  \item to create a working PType, we need to write the
    \texttt{parse\_tar\_numstring}, \texttt{dump\_tar\_numstring} and
    the \texttt{value\_of\_numstring} functions
  \item we replace the old \texttt{string} fields by
    \texttt{tar\_numstring} ones
  \end{itemize}

  \medskip

  \textit{As our new PType needs the length argument at parsing and
    dumping time, the argument is specified using [] instead of ()}
\end{frame}

\begin{frame}{TAR v4/5}
  However \texttt{tar3.ml} is bugged since the \texttt{device\_major}
  and \texttt{device\_minor} are filled with zero bytes when the file
  type is not a device.

  \medskip

  Thus, \texttt{parse\_tar\_numstring} fails and the whole
  \texttt{ustar} header is ignored.

  \medskip

  There are two possible fixes:
  \begin{itemize}
  \item \texttt{tar4.ml} defines a new custom PType
    \texttt{optional\_tar\_numstring}
  \item \texttt{tar5.ml} creates a union using \texttt{file\_type} as
    the discriminating value
  \end{itemize}
\end{frame}

\begin{frame}{Improvements}
  \texttt{tar6.ml}, \texttt{tar7.ml} and \texttt{tar8.ml} later
  improve the parser:
  \begin{itemize}
  \item better display of strings, by taking into account the trailing
    zeroes (\texttt{tar6.ml})
  \item add a \texttt{try..with} in the \texttt{main} function to
    handle the end of file (\texttt{tar7.ml})
  \item add a list and a \texttt{checkpoint} to handle the end of file
    (alternative solution, \texttt{tar8.ml})
  \end{itemize}
\end{frame}



%%%%%%%%%%%%%%%%%%%%%%%%%%%%%%%%%%%%%%%%%%%%%%%%%%%%%%%%%%%%%%%%%%%%%%%%%
\section{Step by step DNS}

\begin{frame}{Outline}
  \tableofcontents[currentsection]
\end{frame}


\begin{frame}{DNS messages (1/4)}
  The original specification is RFC 1035 (updated by some RFCs)

  \medskip

  Message layout:
  \begin{center}
    \scriptsize

    \begin{tabular}{|r|r|l|}
      \hline
      \multicolumn{1}{|c|}{\bf Offset} &
      \multicolumn{1}{c|}{\bf Len.} &
      \multicolumn{1}{c|}{\bf Description} \\
      \hline

      \hline
        0 &   2 & QId \\
      \hline
        2 &   2 & \textit{flags} \\
      \hline
        4 &   2 & Question count \\
      \hline
        6 &   2 & Answer count \\
      \hline
        8 &   2 & Authority record count \\
      \hline
       10 &   2 & Additional record count \\
      \hline
       12 &   ? & Questions \\
      \hline
        ? &   ? & Answers \\
      \hline
        ? &   ? & Authority records \\
      \hline
        ? &   ? & Additional records \\
      \hline
    \end{tabular}
  \end{center}
\end{frame}

\begin{frame}{DNS messages (2/4)}
  Question format:
  \begin{itemize}
  \item a domain,
  \item a 16-bit \texttt{query\_type}
  \item a 16-bit \texttt{query\_class}
  \end{itemize}

  \medskip

  A domain is a sequence of labels, a label being:
  \begin{itemize}
  \item a string if the two first bits are zeroes, the six next bits
    containing the string length
  \item an empty label (a zero byte) to signal the end of a domain
  \item a pointer to compress the domain, two bytes beginning by
    \texttt{0b11} and followed by a 14-bit offset to retrieve the end
    of the domain in the already parsed message
  \end{itemize}
\end{frame}

\begin{frame}{DNS messages (3/4)}
  Resource Record (RR) format:
  \begin{itemize}
  \item a domain (cf. previous slide)
  \item a 16-bit \texttt{rr\_type}
  \item a 16-bit \texttt{rr\_class}
  \item a 32-bit TTL (time-to-live)
  \item a 16-bit integer representing the size of the data
  \item the RR data
  \end{itemize}

  \medskip

  Here are examples of RR data:
  \begin{itemize}
  \item \texttt{A} RR contains an IPv4 address (32~bits)
  \item \texttt{CNAME} RR is an alias pointing towards a domain
  \item \texttt{MX} RR establishes mail exchanger information (a
    16-bit integer and a domain)
  \end{itemize}
\end{frame}


\begin{frame}{DNS messages (4/4)}
  Here a some possible values of \texttt{query\_type} /
  \texttt{rr\_type}:
  \begin{center}
    \scriptsize
    \begin{tabular}{|>{\tt}c|>{\tt}l|c|}
      \hline
      \multicolumn{1}{|c|}{\bf Value} &
      \multicolumn{1}{c|}{\bf Description} &
      \multicolumn{1}{c|}{\bf Compatible with \texttt{rr\_type}} \\
      \hline

      \hline
      1 & A & yes \\
      \hline
      2 & NS & yes \\
      \hline
      5 & CNAME & yes \\
      \hline
      6 & SOA & yes \\
      \hline
      12 & PTR & yes \\
      \hline
      15 & MX & yes \\
      \hline
      255 & * & - \\
      \hline
    \end{tabular}
  \end{center}

  \medskip

  And some values for \texttt{query\_class} / \texttt{rr\_class}:
  \begin{center}
    \scriptsize
    \begin{tabular}{|>{\tt}c|>{\tt}l|c|}
      \hline
      \multicolumn{1}{|c|}{\bf Value} &
      \multicolumn{1}{c|}{\bf Description} &
      \multicolumn{1}{c|}{\bf Compatible with \texttt{rr\_class}} \\
      \hline

      \hline
      1 & Internet & yes \\
      \hline
      2 & CSNET & yes \\
      \hline
      3 & CHAOS & yes \\
      \hline
      4 & Hesiod & yes \\
      \hline
      255 & * & - \\
      \hline
    \end{tabular}
  \end{center}
\end{frame}


\begin{frame}{DNS: first implementation}
  \texttt{dns-steps/dns1.ml} is a first description of DNS messages:
  \begin{itemize}
  \item two \texttt{enum}s for \texttt{rr\_type} and \texttt{rr\_class}
  \item a custom \texttt{label} PType
  \item a custom \texttt{domain} PType to implement a list of
    \texttt{label}s
  \item two \texttt{struct}s describing questions and RRs
  \item a \texttt{dns\_message} structure to wrap everything up
  \item finally, some piece of code to use the generated
    \texttt{parse\_dns\_message} funcion�
  \end{itemize}
\end{frame}

\begin{frame}{DNS: enriched RRs}
  \texttt{dns-steps/dns2.ml} enriches the \texttt{rdata} field:
  \begin{itemize}
  \item new PType: a \texttt{union} called \texttt{rdata} where the
    discriminator is the RR type
  \item for some RRs, description of the RR data (A, CNAME, MX)
  \item change of the \texttt{rdata} field type from
    \texttt{binstring} to \texttt{rdata(rtype)} (the union just
    defined)
  \end{itemize}
\end{frame}

\begin{frame}{Toward a trivial DNS client}
  \begin{itemize}
  \item \texttt{dns3.ml} consists of a rewrite of \texttt{domain} and
    \texttt{label}
  \item \texttt{dns4.ml} introduces a parameter to the \texttt{domain}
    PType, \texttt{context}, to record and expand the label
    compression at parsing time
  \item \texttt{dns5.ml}, \texttt{dns6.ml} and \texttt{dns7.ml} add
    some code in the \texttt{main} function to send a request to a
    real server, and to print the result
  \end{itemize}
\end{frame}


%%%%%%%%%%%%%%%%%%%%%%%%%%%%%%%%%%%%%%%%%%%%%%%%%%%%%%%%%%%%%%%%%%%%%%%%%
\section{Conclusion}

\begin{frame}{Outline}
  \tableofcontents[currentsection]
\end{frame}


\begin{frame}{Related work}
  \begin{itemize}
  \item Scapy
  \item Hachoir
  \item OCaml \texttt{bitstring} library
  \item NetZob
  \item Bro's binpac language
  \end{itemize}
\end{frame}

\begin{frame}{File formats and network protocols implemented}
  \begin{itemize}
  \item X.509
  \item TLS
  \item MRT+BGP
  \item TAR
  \item PCAP/IP/TCP/UDP (trivial description)
  \item DNS
  \item PNG
  \item PE (work in progress)
  \item ExpROM (work in progress)
  \item Kerberos (work in progress)
  \end{itemize}
\end{frame}


\begin{frame}{Questions ?}
  \vspace*{\stretch{1}}

  \begin{center}
    Thank you for your attention.
  \end{center}

  \vspace*{\stretch{1}}
\end{frame}




\end{document}
