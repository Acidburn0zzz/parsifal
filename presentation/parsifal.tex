\documentclass{beamer}
\usetheme{anssi}
\usepackage[french]{babel}
\usepackage[latin1]{inputenc}
\usepackage[T1]{fontenc}
\usepackage{amsmath,amsthm,amssymb}
\usepackage{colortbl}
\usepackage{eso-pic}
\usepackage{longtable}
\usepackage{hhline}
\usepackage{fancyhdr}
\usepackage{pstricks}
\usepackage{pst-node}
\usepackage{pst-text}
\usepackage{etex}
\usepackage{tikz}
\usepackage{listings}

\usetikzlibrary{arrows,shapes}

\begin{document}

\begin{frame}
  \mytitlepage{Parsifal: �criture rapide de \emph{parsers}}{}
  {Olivier Levillain}{ANSSI}{20 novembre 2012}{fr}
\end{frame}

\usenavigation{yes}{4}{1}
\usepagenumbering{yes}
\setbeamertemplate{headline}[body]
\setbeamertemplate{footline}[body]
\title{Parsifal: �criture rapide de \textit{parsers}}


\begin{frame}{Historique (1/2)}
  \vspace{12mm}

  \begin{itemize}
  \item<1-> Au d�part, de nombreuses donn�es TLS: 10~campagnes de mesure
    sur les r�ponses HTTPS du monde IPv4:
    \begin{itemize}
    \item 1~campagne men�e par Arnaud Ebalard en juillet 2010
    \item 2~campagnes men�es par l'EFF en ao�t et d�cembre 2010
    \item 7~campagnes simultan�es (diff�rents stimuli) men�es par le
      LRP (Arnaud pour la partie mesures) depuis TSP en juillet 2011
    \item Total: 143~Go de donn�es � traiter\\[.5cm]
    \end{itemize}

  \item<2-> L'objectif �tait d'en extraire
    \begin{itemize}
    \item des donn�es pertinentes (messages TLS re�us, certificats
      pr�sent�s)
    \item rapidement
    \item de mani�re robuste
    \end{itemize}
  \end{itemize}

\end{frame}


\begin{frame}{Historique (2/2)}
  \vspace{12mm}

  Pour traiter ce volume de donn�es, plusieurs implantations
  \begin{itemize}
  \item<2-> premier prototype en Python: rapide � �crire, mais lent
  \item<3-> seconde version en C++ (avec \textit{templates} et des objets):
    \begin{itemize}
    \item plus rapide, mais
    \item p�nible � mettre au point (fuites m�moire, \emph{segfaults})
    \item n�cessite trop de code pour ajouter des fonctionnalit�s
    \end{itemize}
  \item<4-> troisi�me version en OCaml (avec un DSL, \textit{Domain Specific Language}):
    \begin{itemize}
    \item rapide et flexible
    \item plus robuste que le pr�c�dent
    \item mais encore beaucoup trop de code � �crire
    \end{itemize}
  \item<5-> quatri�me version en OCaml, en utilisant un pr�-processeur (camlp4):
    \begin{itemize}
    \item tous les indicateurs sont au vert!
    \end{itemize}
  \end{itemize}

\end{frame}


\begin{frame}{Questions ?}
  \vspace{30mm}

  \begin{center}
    Merci de votre attention.
  \end{center}

\end{frame}




\end{document}
